\documentclass[10pt,a4paper]{article}

\usepackage[UTF8]{ctex}
\usepackage{geometry}
\usepackage{amsmath,amssymb,amsfonts,bm}
\usepackage{setspace}
\usepackage[numbers,sort&compress]{natbib}
\usepackage[colorlinks=true,linkcolor=blue,citecolor=blue,urlcolor=blue]{hyperref}
\usepackage[thmmarks]{ntheorem}
\usepackage{booktabs}  % 用于美化表格的线型和间距
\usepackage{array}     % 增强表格列格式控制
\usepackage{tabularx}  % 如果需要自动调整列宽的表格
\usepackage{longtable} % 用于跨页的长表格

\geometry{left=2.3cm,right=2.3cm,top=2.2cm,bottom=2.2cm}
\setstretch{1.25}

\newcommand{\R}{\mathbb{R}}
\newcommand{\norm}[1]{\left\lVert #1 \right\rVert}
\newcommand{\ip}[2]{\left\langle #1, #2 \right\rangle}
\newcommand{\conv}{\ast}
\newcommand{\odotop}{\odot}

\title{\textbf{EAGLE-Spine:Energy-Adaptive Geometry-Loop-free Enhancement for Spine Landmark Detection}}
\author{\textit{For Early-stage Theoretical Derivation and Novelty Check}}
\date{\today}

\begin{document}
\maketitle


% =========================================================
% EAGLE-Spine:Energy-Adaptive Geometry-Loop-free Enhancement for Spine Landmark Detection
% 中文Method草稿(用于理论推导与验证)
% 说明:不突出“单任务/多任务”,以“关键点为主、可选几何头”为统一框架描述
% =========================================================

\section{方法(Method)}

\subsection{方法概述与命名}
我们提出 \textbf{EAGLE-Spine}(Energy-Adaptive Geometry-Loop-free Enhancement),面向脊柱X-ray中的椎体角点/关键点定位与链式几何一致性建模。EAGLE-Spine 的核心思想是:在\textbf{动态采样算子}与\textbf{结构连续性正则}两条线上引入\emph{尺度无关(scale-free)}与\emph{环路无关(loop-free)}的约束机制,从而在低质、遮挡、端椎体困难等场景下提升关键点稳定性,并降低由过平滑引起的病理偏差。

整体上,EAGLE-Spine 包含两项关键机制:
\begin{itemize}
    \item \textbf{SREG(Scale-free Relative Energy Gating)}:无量纲相对能量门控,基于中心线局部转角能量构造门控系数,调节连续性正则强度,避免病理奇点被抹平;
    \item \textbf{LODA-Conv (loop-free orthogonal deformable alignment conv)}:环路无关的正交散布约束可变形卷积,通过法向散布先验引导采样覆盖椎体块状结构,同时切断由预测几何量引入的梯度循环风险。
\end{itemize}

\subsection{问题定义与符号约定}
给定一张脊柱正位X-ray图像 $I\in\mathbb{R}^{H\times W}$,我们预测 $N$ 个椎体的关键点集合。每个椎体标注 $Q$ 个关键点(典型取 $Q=4$ 个角点),总关键点数 $M=NQ$。第 $i$ 个椎体的关键点为
\begin{equation}
\mathcal{K}_i=\{\mathbf{k}^{(1)}_i,\mathbf{k}^{(2)}_i,\dots,\mathbf{k}^{(Q)}_i\},\quad 
\mathbf{k}^{(m)}_i=(x^{(m)}_i,y^{(m)}_i)^\top.
\end{equation}
网络输出关键点热力图 $\hat{H}\in\mathbb{R}^{M\times H'\times W'}$。必要时,可选地输出轻量几何参数(例如中心点置信度或形状比),但EAGLE-Spine 的核心机制不依赖“多任务设计”为前提。

\subsection{整体架构}
EAGLE-Spine 由 Backbone、LODA-Conv 增强块与 Landmark Decoder 组成:
\begin{equation}
\{F^s\}_{s=1}^S = \Phi(I), \qquad
\{\hat{F}^s\}_{s=1}^S = \Gamma(\{F^s\}; \text{LODA-Conv}), \qquad
\hat{H} = \Psi(\{\hat{F}^s\}).
\end{equation}
其中 $\Phi$ 为任意主流特征提取网络(CNN/Transformer/Hybrid),$\Gamma$ 在若干关键层插入 LODA-Conv,$\Psi$ 为热力图解码器。

\subsection{关键点热力图预测与主任务监督}
\subsubsection{关键点解码}
从第 $m$ 个关键点热力图 $\hat{H}_m(\mathbf{p})$ 解码坐标。可用 soft-argmax:
\begin{equation}
\hat{\mathbf{k}}_m=\sum_{\mathbf{p}} \mathbf{p}\cdot \mathrm{softmax}\big(\beta \hat{H}_m(\mathbf{p})\big),
\end{equation}
其中 $\beta$ 为温度系数;亦可使用峰值提取 $\arg\max_{\mathbf{p}}\hat{H}_m(\mathbf{p})$。

\subsubsection{主任务损失}
GT热力图 $H^{gt}_m$ 由二维高斯生成,主任务损失采用MSE(可替换为focal heatmap):
\begin{equation}
\mathcal{L}_{hm}=\sum_{m=1}^{M}\|\hat{H}_m - H_m^{gt}\|_2^2.
\end{equation}

\subsection{几何派生:从关键点到中心线(用于SREG与LODA-Conv约束)}
\subsubsection{椎体中心点序列}
定义第 $i$ 个椎体中心点为关键点均值:
\begin{equation}
\hat{\mathbf{c}}_i=\frac{1}{Q}\sum_{m=1}^{Q}\mathrm{sg}\big(\hat{\mathbf{k}}^{(m)}_i\big),
\label{eq:center}
\end{equation}
其中 $\mathrm{sg}(\cdot)$ 表示 stop-gradient,用于在训练早期提高稳定性并避免潜在梯度环路。该操作可在消融实验中讨论其必要性。

\subsubsection{切向与法向向量}
相邻方向向量:
\begin{equation}
\hat{\mathbf{u}}_{i,i+1}=\frac{\hat{\mathbf{c}}_{i+1}-\hat{\mathbf{c}}_i}{\|\hat{\mathbf{c}}_{i+1}-\hat{\mathbf{c}}_i\|_2+\epsilon}.
\end{equation}
局部切向与法向定义为:
\begin{equation}
\hat{\mathbf{t}}_i=\mathrm{sg}\left(\frac{\hat{\mathbf{u}}_{i-1,i}+\hat{\mathbf{u}}_{i,i+1}}{\|\hat{\mathbf{u}}_{i-1,i}+\hat{\mathbf{u}}_{i,i+1}\|_2+\epsilon}\right),\quad
\hat{\mathbf{n}}_i=
\begin{bmatrix}
0 & -1\\
1 & 0
\end{bmatrix}\hat{\mathbf{t}}_i.
\label{eq:tn}
\end{equation}

\subsubsection{无量纲形状比(宽高比)}
为避免尺度不稳,引入无量纲形状比 $r_i$。以四角点为例,定义近似宽度与高度:
\begin{align}
\hat{w}_i &= \frac{1}{2}\Big(\|\hat{\mathbf{k}}^{(R1)}_i-\hat{\mathbf{k}}^{(L1)}_i\|_2 + \|\hat{\mathbf{k}}^{(R2)}_i-\hat{\mathbf{k}}^{(L2)}_i\|_2\Big),\\
\hat{h}_i &= \frac{1}{2}\Big(\|\hat{\mathbf{k}}^{(U1)}_i-\hat{\mathbf{k}}^{(D1)}_i\|_2 + \|\hat{\mathbf{k}}^{(U2)}_i-\hat{\mathbf{k}}^{(D2)}_i\|_2\Big),\\
r_i &= \mathrm{sg}\left(\frac{\hat{w}_i}{\hat{h}_i+\epsilon}\right).
\label{eq:ratio}
\end{align}
(注:角点索引按数据标注顺序映射。)

\subsection{SREG:尺度无关相对能量门控(Scale-free Relative Energy Gating)}

\subsubsection{无量纲局部转角能量(Scale-free Local Turning Energy)}
我们将脊柱中心线视为由椎体中心点 $\{\hat{\mathbf{c}}_i\}_{i=1}^{N}$ 构成的离散链。定义相邻段的单位方向向量
\begin{equation}
\hat{\mathbf{u}}_{i,i+1}=\frac{\hat{\mathbf{c}}_{i+1}-\hat{\mathbf{c}}_{i}}{\|\hat{\mathbf{c}}_{i+1}-\hat{\mathbf{c}}_{i}\|_2+\epsilon},
\end{equation}
则第 $i$ 个节点处的无量纲转角能量定义为
\begin{equation}
\rho_i
=1-\cos\big(\hat{\mathbf{u}}_{i-1,i},\hat{\mathbf{u}}_{i,i+1}\big)
=1-\frac{\hat{\mathbf{u}}_{i-1,i}^\top \hat{\mathbf{u}}_{i,i+1}}
{\|\hat{\mathbf{u}}_{i-1,i}\|_2\,\|\hat{\mathbf{u}}_{i,i+1}\|_2+\epsilon},
\quad i=2,\dots,N-1.
\label{eq:rho}
\end{equation}
由于 $\rho_i$ 仅由夹角决定,$\rho_i\in[0,2]$,并且对坐标整体缩放 $\hat{\mathbf{c}}_i\mapsto s\hat{\mathbf{c}}_i$($s>0$)保持不变,因此天然具备尺度无关性(scale-free)。

\subsubsection{混合鲁棒尺度归一(MAD + Median)与可学习温度}
仅依赖 $\rho_i$ 的绝对大小仍可能受到个体差异影响:不同患者/不同严重程度的整体弯曲背景不同,导致相同 $\rho_i$ 的语义(“正常弯曲”或“异常突变”)不一致。因此我们对 $\{\rho_i\}$ 进行\textbf{相对化},引入鲁棒的背景尺度 $S$。考虑到仅使用 MAD 在近直样本上可能过小、从而使门控过于敏感,我们采用\textbf{混合鲁棒尺度}:
\begin{align}
\mu_\rho &= \mathrm{median}(\{\rho_i\}) , \\
\mathrm{MAD}_\rho &= \mathrm{median}\big(\{|\rho_i-\mu_\rho|\}\big), \\
S &= \mathrm{MAD}_\rho + \gamma\cdot \mu_\rho + \epsilon,
\label{eq:hybrid_scale}
\end{align}
其中 $\gamma\ge 0$ 为可学习或固定的混合系数。该设计兼具两点优点:(1)对离群突变点鲁棒(MAD),(2)在整体弯曲能量较低时仍提供合理尺度地板(Median项),避免门控过早饱和。

基于相对能量 $\rho_i / S$,我们定义门控函数为
\begin{equation}
Gate_i=\lambda_{\min}+(1-\lambda_{\min})\exp\left(-\frac{\rho_i}{\tau\cdot S}\right),
\label{eq:gate}
\end{equation}
其中 $\tau>0$ 为可学习温度(temperature),控制门控对突变的敏感度;$\lambda_{\min}\in(0,1)$ 为门控下界,防止在极端突变处 $Gate_i\rightarrow 0$ 导致梯度断流或训练不稳定。
(在我们的经验设置中,$\tau$ 的初始化取 $2$ 能获得更连续、不过度“硬阈值化”的门控曲线。)

\subsubsection{门控用于连续性正则的自适应加权(Adaptive Regularization Weighting)}
为避免门控直接抑制特征通道带来的优化风险,我们默认将 $Gate_i$ 作为\textbf{连续性/自然性正则}的权重,对离散链的弯曲能量进行自适应惩罚:
\begin{equation}
\mathcal{L}_{cont}=\sum_{i=2}^{N-1} Gate_i\cdot \rho_i.
\label{eq:lcont}
\end{equation}
直觉上,若局部转角能量与整体背景一致(平滑正常段),则 $Gate_i$ 较大,从而增强结构一致性约束以提升关键点稳定性;若出现显著突变(潜在病理奇点),则 $Gate_i$ 自动减小,从而减弱平滑约束、保留真实结构突变,降低由过平滑导致的病理偏差(over-smoothing bias)。

\subsection{LODA-Conv:环路无关的正交散布约束可变形卷积}

\subsubsection{可变形卷积形式}
给定输入特征图 $x$,可变形卷积在输出位置 $\mathbf{p}_0$ 的计算形式为
\begin{equation}
y(\mathbf{p}_0)=\sum_{k=1}^{K} w_k\cdot x\left(\mathbf{p}_0+\mathbf{p}_k+\Delta\mathbf{p}_k(\mathbf{p}_0)\right),
\label{eq:dcn}
\end{equation}
其中 $\mathbf{p}_k$ 为规则采样网格偏移(如 $3\times 3$ 对应 $K=9$),$\Delta\mathbf{p}_k(\mathbf{p}_0)$ 为在位置 $\mathbf{p}_0$ 处预测得到的可学习偏移(以双线性插值在连续坐标取样)。

\subsubsection{解耦偏移预测(避免几何-语义前向耦合与闭环误差)}
偏移由轻量预测网络 $g(\cdot)$ 仅从局部语义特征预测:
\begin{equation}
\Delta\mathbf{P}(\mathbf{p}_0)=g\big(F(\mathbf{p}_0)\big)\in\mathbb{R}^{K\times 2},
\label{eq:offset}
\end{equation}
其中 $F(\mathbf{p}_0)$ 为 $\mathbf{p}_0$ 处的局部特征。关键点在于:$g(\cdot)$ 的输入\textbf{不包含}当前迭代的中心线、切线或关键点预测,从而避免形成
\[
\text{几何预测}\rightarrow \text{采样方向/位置}\rightarrow \text{特征分布}\rightarrow \text{几何预测}
\]
的前向耦合链路,降低闭环误差积累导致的特征坍缩风险。几何信息仅用于构造约束项(见下),并默认采用 stop-gradient($\mathrm{sg}(\cdot)$)以进一步抑制梯度循环。

\subsubsection{正交散布约束(Orthogonal Dispersion Constraint):仅约束法向覆盖,保留切向自由度}
在遮挡/低质/端椎体场景中,传统可变形采样易被细长的干扰边缘(如肋骨)吸引,导致采样域漂移。为引导采样覆盖椎体\textbf{块状结构}而非沿中心线“一维排队”,我们不对偏移的切向方向作硬对齐,而是约束其在法向方向的\textbf{散布(dispersion)}满足形状先验。

对第 $i$ 个椎体,给定其切向/法向单位向量 $\hat{\mathbf{t}}_i,\hat{\mathbf{n}}_i$(由中心线派生,并默认 $\mathrm{sg}$),将偏移投影到法/切向:
\begin{equation}
d_k=\Delta\mathbf{p}_k\cdot \hat{\mathbf{n}}_i,\quad
s_k=\Delta\mathbf{p}_k\cdot \hat{\mathbf{t}}_i.
\end{equation}
以方差刻画散布幅度:
\begin{equation}
\mathcal{D}^n_i=\sqrt{\mathrm{Var}_k(d_k)+\epsilon},\quad
\mathcal{D}^t_i=\sqrt{\mathrm{Var}_k(s_k)+\epsilon}.
\end{equation}
其中 $\mathcal{D}^n_i$ 表示采样域在法向的“横向覆盖宽度”,$\mathcal{D}^t_i$ 表示切向上的延展。结合由关键点派生的无量纲形状比 $r_i$(式\ref{eq:ratio}),我们约束散布比值为:
\begin{equation}
\mathcal{L}_{ortho}=\sum_{i=1}^{N}\left\|
\frac{\mathcal{D}^n_i}{\mathcal{D}^t_i+\epsilon}-\beta\cdot r_i
\right\|_2^2,
\label{eq:lortho}
\end{equation}
其中 $\beta\ge 0$ 为可学习标量(用于对齐数值尺度)。该约束具有三个关键特性:
\begin{itemize}
    \item \textbf{形状对齐(Block-wise Coverage)}:通过控制法向散布,使采样点在椎体横向方向展开,抑制采样退化为细长一维排列,从而在遮挡/端椎体处更不易漂移;
    \item \textbf{尺度鲁棒(Scale-free Ratio)}:约束量为无量纲比值,天然抵抗分辨率/缩放变化;
    \item \textbf{环路抑制(Loop-free Training)}:几何量仅用于约束空间并默认 $\mathrm{sg}$,而偏移预测 $g(\cdot)$ 不接收几何输入,有效切断几何--采样--特征的正反馈回路。
\end{itemize}

\paragraph{实现细节(用于稳定训练的局部采样)}
为避免为全图建立椎体归属映射,我们在训练中仅在椎体中心点附近的少量采样位置评估式(\ref{eq:lortho}):将中心点投影到特征图坐标,在其邻域(如 $3\times 3$ patch)通过双线性方式读取对应位置的偏移 $\Delta\mathbf{P}$,并在这些位置上汇聚估计 $\mathcal{D}^n_i,\mathcal{D}^t_i$。该做法既稳定又开销可控。

\subsubsection{弱防漂移正则}
为避免偏移无界增大导致采样点远离目标区域,引入弱正则:
\begin{equation}
\mathcal{L}_{drift}=\sum_{k=1}^{K}\|\Delta\mathbf{p}_k\|_2^2,
\label{eq:ldrift}
\end{equation}
实际训练中设置较小权重以避免抑制可变形表达能力。


\subsection{总目标函数}
综合主任务监督与两项机制的正则项,整体损失为:
\begin{equation}
\mathcal{L}=
\lambda_{hm}\mathcal{L}_{hm}
+\lambda_{ortho}\mathcal{L}_{ortho}
+\lambda_{drift}\mathcal{L}_{drift}
+\lambda_{cont}\mathcal{L}_{cont}.
\label{eq:overall}
\end{equation}
其中 $\lambda_{\cdot}$ 为权重超参数;$\tau$ 与(可选)$\beta$ 可通过梯度学习。

\subsection{训练策略(稳定性优先)}
为避免几何约束在训练早期扰动关键点学习,我们采用分阶段启用策略:
\begin{itemize}
    \item \textbf{阶段A:基础学习} 仅优化 $\mathcal{L}_{hm}$,获得稳定的关键点响应;
    \item \textbf{阶段B:启用LODA-Conv} 插入可变形采样并开启 $\mathcal{L}_{ortho}$ 与弱 $\mathcal{L}_{drift}$;
    \item \textbf{阶段C:启用SREG} 计算 $\rho_i$ 与 $Gate_i$,开启加权连续性正则 $\mathcal{L}_{cont}$,并学习 $\tau$。
\end{itemize}
在所有阶段,几何派生量默认 stop-gradient;可在消融中对比 stop-gradient、teacher-EMA 或完全可导设置对稳定性与性能的影响。

\subsection{推理阶段}
推理时输出关键点热力图并解码得到 $\hat{\mathbf{k}}^{(m)}_i$。中心线、端板方向与Cobb角等临床指标在后处理中由关键点几何关系计算得到(本文草稿阶段暂不展开具体Cobb计算协议)。

\subsection{建议的关键消融(用于理论验证与可复现实验)}
\begin{enumerate}
    \item 基线:仅 $\mathcal{L}_{hm}$(无LODA-Conv,无SREG);
    \item + 可变形采样(无约束):加入式(\ref{eq:dcn})但不加 $\mathcal{L}_{ortho}$;
    \item + LODA-Conv:加入 $\mathcal{L}_{ortho}$,对比 $\mathrm{sg}(\cdot)$ 是否启用;
    \item 先验形式:散布比值约束(式\ref{eq:lortho}) vs 绝对宽度(对照用);
    \item + SREG:加入 $\mathcal{L}_{cont}$,对比尺度估计 MA vs MAD、$\lambda_{\min}$ 取值;
    \item 稳定性:记录训练早期梯度范数、offset分布统计(均值/方差/极值)、loss爆炸率;
    \item 压力测试:端椎体、遮挡、跨分辨率缩放扰动(scale stress test)。
\end{enumerate}
\section{算子级验证与几何对齐分析}

为了验证所提出的 LODA-Conv 算子在捕获脊柱解剖结构特征时的有效性,我们设计了针对算子采样拓扑的统计验证实验。本节旨在定量评估算子在缺乏全局语义监督的情况下,仅依靠几何约束实现“法向扩张、切向稳定”的结构感知能力。

\subsection{评估指标定义}

我们定义以下三个核心指标来刻画算子的拓扑演化特性与几何忠实度:

\begin{enumerate}
    \item \textbf{法向/切向离散度 ($D_n, D_t$)}:用于度量采样点偏移向量 $\Delta \mathbf{p}$ 在椎体局部坐标系下的分布范围:
    \begin{equation}
        D_n = \mathrm{std}(|\Delta \mathbf{p} \cdot \mathbf{n}|), \quad D_t = \mathrm{std}(|\Delta \mathbf{p} \cdot \mathbf{t}|)
    \end{equation}
    其中 $\mathbf{n}$ 与 $\mathbf{t}$ 分别代表椎体的局部法向与切向单位向量。
    
    \item \textbf{拓扑演化比率 (NER 指标)}:定义为法向与切向离散度的比值 $R = D_n / D_t$。该指标反映了采样域的形状演化趋势。理想情况下,算子应表现出显著的各向异性,即 $D_n$ 快速上升以覆盖椎体横向结构,而 $D_t$ 保持低位以维持链式连续性。
    
    \item \textbf{形状对齐误差 (Shape Alignment Error, SAE)}:用于衡量采样拓扑与真实解剖比例的契合程度:
    \begin{equation}
        \mathrm{SAE} = \left| R - r_i \right| = \left| \frac{D_n}{D_t} - r_i \right|
    \end{equation}
    其中 $r_i$ 为目标椎体的解剖宽高比。$\mathrm{SAE}$ 数值越小,表明算子的采样分布越符合解剖学先验。
\end{enumerate}

\subsection{统计结果与对比分析}

我们基于配对置换检验 (Paired Permutation Test) 对标准卷积 (Standard Conv)、无约束可变形卷积 (Unconstrained DCN) 以及 LODA-Conv 进行了 500 轮统计迭代验证。实验结果如表 \ref{tab:loda_results} 所示。

\begin{table}[htbp]
    \centering
    \caption{不同算子在几何对齐精度 (SAE) 上的统计对比 (数值越低越好)}
    \label{tab:loda_results}
    \begin{tabular}{lccc}
        \toprule
        评估方法 & 平均误差 (Mean) & 95\% 置信区间 (95\% CI) & 显著性 $P$ 值 (vs. LODA) \\
        \midrule
        Standard Conv & 2.0645 & (1.9494, 2.1892) & 0.0005 \\
        DCN (Unconstrained) & 1.0953 & (0.9500, 1.2518) & 0.0005 \\
        \textbf{LODA-Conv (Ours)} & \textbf{0.7732} & \textbf{(0.6730, 0.8829)} & -- \\
        \bottomrule
    \end{tabular}
\end{table}

\subsection{实验结论与讨论}

通过对上述定量指标的分析,可以得出以下结论:

\begin{itemize}
    \item \textbf{卓越的几何忠实度}:LODA-Conv 的平均 SAE 仅为 0.7732,相较于 Standard Conv 显著降低了约 62.5\%,相较于无约束 DCN 降低了 29.4\%。这表明引入的正交散布约束项 $\mathcal{L}_{\text{ortho}}$ 能够有效驱动 $R$ 向 $r_i$ 靠拢,强制采样拓扑与椎体块状结构对齐。
    
    \item \textbf{统计学稳健性}:LODA-Conv 的 95\% 置信区间不仅显著低于对照组,且分布更加集中。配合极显著的 $P$ 值 ($p=0.0005$),证明了性能提升在不同椎体形态下具有高度的一致性与可复现性。
    
    \item \textbf{各向异性约束的优越性}:NER 指标的演化趋势证实了 $D_n$ 与 $D_t$ 实现了有效解耦。算子在法向上表现出的主动扩张能力,确保了模型能够捕获到由于遮挡或低对比度而易丢失的椎体边缘特征,从底层解决了传统可变形卷积在医学影像中易发生的“采样坍缩”问题。
\end{itemize}


\end{document}